\documentclass[a4paper, 12pt]{article}
\usepackage[brazil]{babel}
\usepackage[utf8]{inputenc}
\usepackage{indentfirst}
\usepackage{float}
\usepackage{listings}
\usepackage{color}

\restylefloat{table}

\definecolor{mygreen}{rgb}{0,0.6,0}
\definecolor{mygray}{rgb}{0.5,0.5,0.5}
\definecolor{mymauve}{rgb}{0.58,0,0.82}

\lstset{ %
  backgroundcolor=\color{white},   % choose the background color; you must add \usepackage{color} or \usepackage{xcolor}; should come as last argument
  basicstyle=\footnotesize,        % the size of the fonts that are used for the code
  breakatwhitespace=false,         % sets if automatic breaks should only happen at whitespace
  breaklines=true,                 % sets automatic line breaking
  captionpos=b,                    % sets the caption-position to bottom
  commentstyle=\color{mygreen},    % comment style
  deletekeywords={...},            % if you want to delete keywords from the given language
  escapeinside={\%*}{*)},          % if you want to add LaTeX within your code
  extendedchars=true,              % lets you use non-ASCII characters; for 8-bits encodings only, does not work with UTF-8
  frame=single,	                   % adds a frame around the code
  keepspaces=true,                 % keeps spaces in text, useful for keeping indentation of code (possibly needs columns=flexible)
  keywordstyle=\color{blue},       % keyword style
  language=Octave,                 % the language of the code
  morekeywords={*,...},           % if you want to add more keywords to the set
  numbers=left,                    % where to put the line-numbers; possible values are (none, left, right)
  numbersep=5pt,                   % how far the line-numbers are from the code
  numberstyle=\tiny\color{mygray}, % the style that is used for the line-numbers
  rulecolor=\color{black},         % if not set, the frame-color may be changed on line-breaks within not-black text (e.g. comments (green here))
  showspaces=false,                % show spaces everywhere adding particular underscores; it overrides 'showstringspaces'
  showstringspaces=false,          % underline spaces within strings only
  showtabs=false,                  % show tabs within strings adding particular underscores
  stepnumber=2,                    % the step between two line-numbers. If it's 1, each line will be numbered
  stringstyle=\color{mymauve},     % string literal style
  tabsize=2,	                   % sets default tabsize to 2 spaces
  title=\lstname                   % show the filename of files included with \lstinputlisting; also try caption instead of title
}

\begin{document}
  \begin{center}
      \vspace*{0.2cm}

      \LARGE
      \textbf{Algoritmos genéticos para resolução de problemas de criptoaritmética.}

      \vspace{0.5cm}

      Disciplina de Inteligência Computacional - GBC073

      \vspace{6cm}

      \Large
      Gustavo Rezende Silva - 11311EMT018\\
      Professora Dra. Gina Maira Barbosa de Oliveira\\

      \vspace{7cm}



      \Large
      Universidade Federal de Uberlândia\\

      17 de maio de 2017

  \end{center}
  \pagenumbering{gobble}
  \newpage
    \pagenumbering{roman}
    \tableofcontents

  \newpage

  \pagenumbering{arabic}
  \section{Objetivo}
  Este trabalho tem como objetivo a implementação de um algoritimo genético (AG) para
  resolução de problemas de criptoaritimética. E ainda, possibilitar que o aluno
  entenda a através da prática o efeito dos diversos parâmetros e diferentes
  avaliações nos problemas envolvendo AGs.


  \section{Desenvolvimento}

  \subsection{Etapa 1}
    Nesta fase o AG foi configurado com algumas carecterísticas fixas,
    e outras que foram alteradas, totalizando 8 diferentes configurações,
    e seus resultados comparados. Estas se encontram a seguir:

    Fixas:
    \begin{itemize}
    \item População de 100 indivíduos;
    \item Cross Over de 80\% e cíclico;
    \item 200 gerações;
    \item Mutação deterministica.
    \end{itemize}

    Variáveis:
    \begin{itemize}
    \item Sorteio: Tour 3 e Roleta;
    \item Método de Substituição: Melhores entre pais e filhos (MPF) e Elitismo;
    \item Taxa de mutação: 10\% e 20\%.
    \end{itemize}

    Para cada configuração o programa foi executado 1000 vezes e sua taxa de
     convergência e tempo de execução foram anotados. Os resultados encontrados
     se encontram na Tabela \ref{tab:etapa1}.

    \begin{table}[h]
      \centering
      \begin{tabular}{|c|c|c|c|c|c|}
        \hline
        N & Sorteio & Sub & Mutação & Convergência & Tempo \\
        \hline
        1 & Tour 3 & MPF & 10\% & 43,6\% & 45,58s \\
        \hline
        2 & Tour 3 & ELIT & 10\% & 46,2\% & 5,13s \\
        \hline
        3 & Tour 3 & MPF & 20\% & 43,5\% & 50,39s \\
        \hline
        4 & Tour 3 & ELIT & 20\% & 54,1\% & 5,8s \\
        \hline
        5 & Roleta & MPF & 10\% & 40,2\% & 46,47s \\
        \hline
        6 & Roleta & ELIT & 10\% & 69,8\% & 9,12s \\
        \hline
        7 & Roleta & MPF & 20\% & 42,4\% & 47,15s \\
        \hline
        8 & Roleta & ELIT & 20\% & 82,6\% & 10,33s \\
        \hline
      \end{tabular}
      \caption{Resultados da primeira etapa}
      \label{tab:etapa1}
    \end{table}

  Observando os resultados das configurações propostas a que se mostrou mais
  eficiente foi a numero oito (Tabela \ref{tab:etapa1}). Uma vez que a taxa de convergência é a
  ma mais alta e seu tempo de execução é relativamente baixo comparado com os demais.

  \subsection{Etapa 2}
  A segunda etapa teve como objetivo alterar os parâmetros, fixos ou variáveis,
  da primeira com o intuito de encontrar um resultado melhor. Neste trabalho
  a mutação deterministica foi trocada pela de probabilidade real.

  O programa foi executado 1000 vezes para cada nova configuração
  e seus resultados se encontram na Tabela \ref{tab:etapa2}.

  \begin{table}[h]
    \centering
    \begin{tabular}{|c|c|c|c|c|c|c|c|}
      \hline
      N & População & Gerações & Sorteio & Sub & Mutação & Convergência & Tempo  \\
      \hline
      1 & 100 & 200 & Roleta & ELIT & 20\% & 90,1\% & 10,57s \\
      \hline
      2 & 100 & 200 & Roleta & ELIT & 25\% & 94,5\% & 10,86s \\
      \hline
      3 & 100 & 200 & Roleta & ELIT & 50\% & 98,3\% & 12,29s \\
      \hline
      4 & 200 & 200 & Roleta & ELIT & 50\% & 100,0\% & 51,92s \\
      \hline
      5 & 100 & 500 & Roleta & ELIT & 20\% & 97,0\% & 26,09s \\
      \hline
      6 & 150 & 250 & Roleta & ELIT & 20\% & 98,2\% & 28,19s \\
      \hline
      7 & 100 & 250 & Roleta & ELIT & 25\% & 95,1\% & 13,7s \\
      \hline
      8 & 100 & 200 & TOUR 3 & ELIT & 50\% & 86,5\% & 8,23s \\
      \hline
      9 & 200 & 200 & TOUR 3 & ELIT & 50\% & 94,6\% & 32.16s \\
      \hline
      10 & 200 & 200 & TOUR 5 & MPF & 20\% & 85,5\% & 172,87s \\
      \hline
    \end{tabular}
    \caption{Resultados da segunda etapa}
    \label{tab:etapa2}
  \end{table}

  Ao analisar os resultados apresentados na Tabela \ref{tab:etapa2} percebemos
  que a configuração de número 4 obteve 100\% de taxa de convergência, porém,
  seu tempo de execução é 6,3 vezes o tempo do arranjo 8 que obteve 86,5\%, ou seja,
  pouco aumento de convergência para muito tempo de execução.

  Comparando o resultado 8 com os demais percebe-se que a disposição 2 é apenas
  1,3 vezes mais demorada e sua taxa de convergência é 8\% maior, apresentando assim
  uma boa relação entre tempo de execução e convergência, por isto foi a configuração
  escolhida.
  \subsection{Etapa 3}
  Nesta fase a melhor configuração da Etapa 2, número 2 da Tabela \ref{tab:etapa2},
  foi utilizada para resolver novos problemas, sendo eles:
  \begin{enumerate}
  \item SEND + MORE = MONEY;
  \item EAT + THAT = APPLE;
  \item CROSS + ROADS = DANGER;
  \item COCA + COLA = OASIS;
  \item DONALD + GERALD = ROBERT.
  \end{enumerate}

Os resultados encontrados não foram satisfatórios e podem ser encontrados na
Tabela \ref{tab:etapa3}.

  \begin{table}[h]
    \centering
    \begin{tabular}{|c|c|c|}
      \hline
      Problema & Convergência & Tempo \\
      \hline
      2-EAT & 26,2\% & 10,53s \\
      \hline
      3-CROSS & 1,0\% & 10,56s \\
      \hline
      4-COCA & 9,5\% & 10,34s \\
      \hline
      5-DONALD & 5,1\% & 10,31s \\
      \hline
    \end{tabular}
    \caption{Resultados da terceira etapa}
    \label{tab:etapa3}
  \end{table}

Com o intuito de melhorar os resultados testou-se as funções de avaliação a seguir:

  \begin{equation}
    Av = (\prod_{i=1}^{N} 1+\mid R_i - E_i\mid) -1
    \label{eq:avprod}
  \end{equation}
  \begin{equation}
    Av = \sum_{i=1}^{N} \mid R_i - E_i\mid
    \label{eq:avdifind}
  \end{equation}
Onde:
  \begin{description}
    \item[\(Av\)] = Avaliação
    \item[\(R_i\)] = i-esima letra do resultado da soma
    \item[\(E_i\)] = i-esima letra do resultado esperado
  \end{description}

Os resultados encontrados utilizando estas avaliações com combinações diferentes
dos parâmetros estão listados nas Tabelas \ref{tab:send} até \ref{tab:donald}.

  \begin{table}[H]
    \centering
    \begin{tabular}{|c|c|c|c|c|c|c|c|c|}
      \hline
      N & Av & Pop & Gerações & Sorteio & Sub & Mut. & Conver. & Tempo  \\
      \hline
      1& 1 & 100 & 200 & Roleta & ELIT & 50\% &72,8\% & 12,42s \\
      \hline
      2& 1 & 100 & 200 & Roleta & ELIT & 90\% &87,2\% & 12,74s \\
      \hline
      3& 1 & 100 & 200 & Tour 1 & ELIT & 50\% &71,0\% & 9,76s \\
      \hline
      4& 1 & 100 & 200 & Tour 1 & ELIT & 90\% &86,1\% & 10,44s \\
      \hline
      5& 1 & 150 & 150 & Tour 1 & ELIT & 100\% &91,8\% & 13,8s \\
      \hline
      6& 1 & 300 & 30 & Tour 2 & ELIT & 90\% &96,0\% & 10,38s \\
      \hline
      7& 2 & 100 & 200 & Roleta & ELIT & 50\% &73,0\% & 9,83s \\
      \hline
      8& 2 & 200 & 100 & Roleta & ELIT & 50\% &80,1\% & 15,36s \\
      \hline
      9& 2 & 200 & 100 & Tour 1 & ELIT & 50\% &82,6\% & 11,62s \\
      \hline
      10& 2 & 200 & 90 & Tour 1 & ELIT & 90\% &91,6\% & 10,6s \\
      \hline
    \end{tabular}
    \caption{Novas avaliações para SEND + MORE = MONEY}
    \label{tab:send}
  \end{table}

  Para o problema SEND + MORE = MONEY o melhor resultado foi o número 6 de acordo
  com a Tabela \ref{tab:send}. Isto se deve ao fato de possuir a melhor taxa de
  convergência e tempo de execução inferior aos que possuem convergência semelhante.

  \begin{table}[H]
    \centering
    \begin{tabular}{|c|c|c|c|c|c|c|c|c|}
      \hline
      N & Av & Pop & Gerações & Sorteio & Sub & Mut. & Conver. & Tempo  \\
      \hline
      1 & 1 & 100 & 200 & Roleta & ELIT & 25\% &19,2\% & 11,04s \\
      \hline
      2 & 1 & 100 & 200 & Roleta & ELIT & 50\% &21,3\% & 13,84s \\
      \hline
      3 & 1 & 100 & 200 & Roleta & ELIT & 90\% &28,6\% & 12,65s \\
      \hline
      4 & 1 & 100 & 200 & Tour 1 & ELIT & 50\% &23,01\% & 9,62s \\
      \hline
      5 & 1 & 100 & 200 & Tour 1 & ELIT & 90\% &30,1\% & 10,11s \\
      \hline
      6 & 1 & 150 & 130 & Tour 1 & ELIT & 90\% &43,6\% & 14,16s \\
      \hline
      7 & 1 & 300 & 30 & Tour 2 & ELIT & 90\% &33,1\% &10,21s \\
      \hline
      8 & 2 & 200 & 90 & Roleta & ELIT & 50\% &24,8\% & 14,55s \\
      \hline
      9 & 2 & 200 & 90 & Tour 1 & ELIT & 90\% &30,8\% & 10,75s \\
      \hline
      10& 2 & 150 & 150 & Tour 1 & ELIT & 90\% &29,9\% & 11,79s \\
      \hline
    \end{tabular}
    \caption{Novas avaliações para EAT + THAT = APPLE}
    \label{tab:eat}
  \end{table}

  No problema EAT + THAT = APPLE o resultado com maior taxa de convergência foi
  o sexto de acordo com a Tabela \ref{tab:eat}, porém,  seu tempo de execução
  foi elevado em comparação aos demais. Entretanto, existe
  uma difrença de 10\% de convergência para o segundo maior o que justifica a
  escolha do número 6.

  \begin{table}[H]
    \centering
    \begin{tabular}{|c|c|c|c|c|c|c|c|c|}
      \hline
      N & Av & Pop & Gerações & Sorteio & Sub & Mut. & Conver. & Tempo  \\
      \hline
      1 & 1 & 100 & 200 & Roleta & ELIT & 50\% &4,0\% & 14,10s \\
      \hline
      2&1 & 100 & 200 & Tour 2 & ELIT & 50\% &5,2\% & 10,31s \\
      \hline
      3&1 & 240 & 55 & Tour 2 & ELIT & 90\% &11,7\% & 13,95s \\
      \hline
      4&1 & 240 & 55 & Tour 3 & ELIT & 90\% &10,3\% & 12,86s \\
      \hline
      5&1 & 300 & 30 & Tour 2 & ELIT & 90\% &12,1\% & 11,49s \\
      \hline
      6&2 & 100 & 200 & Roleta & ELIT & 25\% &2,4\% & 9,57s \\
      \hline
      7&2 & 100 & 200 & Roleta & ELIT & 50\% &5,4\% & 9,88s \\
      \hline
      8&2 & 150 & 150 & Roleta & ELIT & 90\% &6,8\% & 15,15s \\
      \hline
      9&2 & 100 & 200 & Tour 1 & ELIT & 25\% &3,5\% & 7,66s \\
      \hline
      10&2 & 100 & 200 & Tour 1 & ELIT & 25\% &5,4\% & 7,91s \\
      \hline
    \end{tabular}
    \caption{Novas avaliações para CROSS + ROADS = DANGER}
    \label{tab:cross}
  \end{table}

  Na criptoaritmética de CROSS + ROADS = DANGER o resultado melhor avaliado
  foi o número 5, Tabela \ref{tab:cross}, uma vez que possui a melhor taxa de
  convergência e tempo de execução próximo do original.

  \begin{table}[H]
    \centering
    \begin{tabular}{|c|c|c|c|c|c|c|c|c|}
      \hline
      N&Av & Pop & Gerações & Sorteio & Sub & Mut. & Conver. & Tempo  \\
      \hline
      1&1& 100 & 200 & Tour 1 & ELIT & 50\% &45,7\% & 11,19s\\
      \hline
      2&1& 255 & 60 & Tour 2 & ELIT & 90\% &77,5\% & 14,32s\\
      \hline
      3&1& 300 & 30 & Tour 2 & ELIT & 90\% &71,0\% & 9,88s\\
      \hline
      4&2& 100 & 200 & Roleta & ELIT & 50\% &44,3\% & 10,06s\\
      \hline
      5&2& 100 & 200 & Roleta & ELIT & 90\% &57,4\% & 10,32s\\
      \hline
      6&2& 120 & 120 & Roleta & ELIT & 90\% &60,4\% & 8,15s\\
      \hline
      7&2& 100 & 200 & Tour 1 & ELIT & 50\% &45,1\% & 7,79s\\
      \hline
      8&2& 100 & 200 & Tour 1 & ELIT & 90\% &58,4\% & 8,42s\\
      \hline
      9&2& 200 & 100 & Tour 1 & ELIT & 90\% &67,2\% & 12,06s\\
      \hline
      10&2& 200 & 100 & Tour 2 & ELIT & 90\% &73,0\% & 11,44s\\
      \hline
    \end{tabular}
    \caption{Novas avaliações para COCA + COLA = OASIS}
    \label{tab:coca}
  \end{table}

  No problema COCA + COLA = OASIS obteve-se como melhor resultado o número 3,
  Tabela \ref{tab:coca}. Esta resposta teve a segunda  maior taxa de convergência,
  mas como seu tempo de execução foi consideravelmente menor para uma diferença
  de apenas 6,5\% na taxa ela foi escolhida.

  \begin{table}[H]
    \centering
    \begin{tabular}{|c|c|c|c|c|c|c|c|c|}
      \hline
      N&Av & Pop & Gerações & Sorteio & Sub & Mut. & Conver. & Tempo  \\
      \hline
      1&1 & 100 & 200 & Roleta & ELIT & 70\% &45,7\% & 14,42s\\
      \hline
      2&1 & 100 & 200 & Roleta & ELIT & 90\% &53,5\% & 14,95s\\
      \hline
      3&1 & 100 & 200 & Tour 1 & ELIT & 50\% &38,0\% & 11,61s\\
      \hline
      4&1 & 240 & 55 & Tour 2 & ELIT & 90\% &76,2\% & 13,37s\\
      \hline
      5&1 & 300 & 30 & Tour 2 & ELIT & 90\% &71,5\% & 12,03s\\
      \hline
      6&2 & 100 & 200 & Roleta & ELIT & 90\% &53,9\% & 10,22s\\
      \hline
      7&2 & 100 & 200 & Tour 1 & ELIT & 50\% &36,1\% & 8,04s\\
      \hline
      8&2 & 100 & 200 & Tour 1 & ELIT & 90\% &49,0\% & 8,57s\\
      \hline
      9&2 & 200 & 100 & Tour 1 & ELIT & 90\% &65,0\% & 12,42s\\
      \hline
      10&2 & 200 & 100 & Tour 3 & ELIT & 90\% &65,5\% & 12,36s\\
      \hline
    \end{tabular}
    \caption{Novas avaliações para DONALD + GERALD = ROBERT}
    \label{tab:donald}
  \end{table}

  Na pergunta DONALD + GERALD = ROBERT a resposta selecionada foi a número 4,
  Tabela \ref{tab:donald}, uma vez que seu indice de convergência foi o maior
  e o tempo de execução foi próximo dos restantes.

  Pode-se perceber que para os diferentes problemas os melhores resultados
  foram dados pela avaliação 1 utilizando o método de sorteio Tour, com diferentes
  tamanhos, substituição Elitista e mutação de 90\%, apresentando variações apenas
  nos outros parâmetros.

  E ainda, ao observar a Tabela \ref{tab:etapa3melhorada} fica evidente que os resultados
  foram melhores do que a avaliação original.

  \begin{table}[H]
    \centering
    \begin{tabular}{|c|c|c|}
      \hline
      Problema & Convergência & Tempo \\
      \hline
      1-SEND & 96,0\% & 10,38s \\
      \hline
      2-EAT & 43,6\% & 14,16s \\
      \hline
      3-CROSS & 12,1\% & 11,49s \\
      \hline
      4-COCA & 71,0\% & 9,88s \\
      \hline
      5-DONALD & 76,2\% & 13,37s \\
      \hline
    \end{tabular}
    \caption{Resultados da terceira etapa novas avaliações}
    \label{tab:etapa3melhorada}
  \end{table}

\section{Referências Bibliográficas}
  \begin{itemize}
    \item Roteiro do trabalho 1 de criptoaritimética. Profa Dra. Gina Maira Barbosa de Oliveira, 2017-1;
    \item Apresentações de computação evolutiva. Profa Dra. Gina Maira Barbosa de Oliveira, 2017-1;
    \item Solving Cryptarithmetic Problems Using Parallel Genetic Algorithm. Reza Abbasian and Masoud Mazloom, 2009.
  \end{itemize}



\section{Apêndice}
  Avaliação origina:
  \lstinputlisting[language=c]{ag.c}

  Avaliação 1:
  \lstinputlisting[language=c, firstline=121, lastline=195]{ag_mult.c}

  Avaliação 2:
  \lstinputlisting[language=c, firstline=121, lastline=191]{ag_difind.c}

\end{document}
