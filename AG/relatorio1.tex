\documentclass[a4paper, 12pt]{article}
\usepackage[brazil]{babel}
\usepackage[utf8]{inputenc}
\usepackage{indentfirst}

\title {Trabalho 1 - Inteligência Computacional}
\date{\today}
\author{Gustavo Rezende Silva}
\begin{document}
  \maketitle
  \pagenumbering{gobble}
  \newpage

  \pagenumbering{arabic}
  \section{Desenvolvimento}

  \subsection{Etapa 1}
    Nesta fase o AG foi configurado com algumas carecterísticas fixas,
    e outras que foram alteradas, totalizando 8 diferentes configurações,
    e seus resultados comparados. Estas se encontram a seguir:

    Fixas:
    \begin{itemize}
    \item População de 100 indivíduos;
    \item Cross Over de 80\% e cíclico;
    \item 200 gerações;
    \item Mutação deterministica.
    \end{itemize}

    Variáveis:
    \begin{itemize}
    \item Sorteio: Tour 3 e Roleta;
    \item Método de Substituição: Melhores entre pais e filhos (MPF) e Elitismo;
    \item Taxa de mutação: 10\% e 20\%.
    \end{itemize}

    Para cada configuração o programa foi executado 1000 vezes e sua taxa de
     convergência e tempo de execução foram anotados. Os resultados encontrados
     se encontram na Tabela \ref{tab:etapa1}.

    \begin{table}[h]
      \centering
      \begin{tabular}{|c|c|c|c|c|c|}
        \hline
        N & Sorteio & Sub & Mutação & Convergência & Tempo \\
        \hline
        1 & Tour 3 & MPF & 10\% & 43,6\% & 45,58s \\
        \hline
        2 & Tour 3 & ELIT & 10\% & 46,2\% & 5,13s \\
        \hline
        3 & Tour 3 & MPF & 20\% & 43,5\% & 50,39s \\
        \hline
        4 & Tour 3 & ELIT & 20\% & 54,1\% & 5,8s \\
        \hline
        5 & Roleta & MPF & 10\% & 40,2\% & 46,47s \\
        \hline
        6 & Roleta & ELIT & 10\% & 69,8\% & 9,12s \\
        \hline
        7 & Roleta & MPF & 20\% & 42,4\% & 47,15s \\
        \hline
        8 & Roleta & ELIT & 20\% & 82,6\% & 10,33s \\
        \hline
      \end{tabular}
      \caption{Resultados da primeira etapa}
      \label{tab:etapa1}
    \end{table}

  Observando os resultados das configurações propostas a que se mostrou mais
  eficiente foi a numero oito (Tabela \ref{tab:etapa1}). Uma vez que a taxa de convergência é a
  ma mais alta e seu tempo de execução é relativamente baixo comparado com os demais.

  \subsection{Etapa 2}
  A segunda etapa teve como objetivo alterar os parâmetros, fixos ou variáveis,
  da primeira com o intuito de encontrar um resultado melhor. Neste trabalho
  a mutação deterministica foi trocada pela de probabilidade real.

  O programa foi executado 1000 vezes para cada nova configuração
  e seus resultados se encontram na Tabela \ref{tab:etapa2}.

  \begin{table}[h]
    \centering
    \begin{tabular}{|c|c|c|c|c|c|c|c|}
      \hline
      N & População & Gerações & Sorteio & Sub & Mutação & Convergência & Tempo  \\
      \hline
      1 & 100 & 200 & Roleta & ELIT & 20\% & 90,1\% & 10,57s \\
      \hline
      2 & 100 & 200 & Roleta & ELIT & 25\% & 94,5\% & 10,86s \\
      \hline
      3 & 100 & 200 & Roleta & ELIT & 50\% & 98,3\% & 12,29s \\
      \hline
      4 & 200 & 200 & Roleta & ELIT & 50\% & 100,0\% & 51,92s \\
      \hline
      5 & 100 & 500 & Roleta & ELIT & 20\% & 97,0\% & 26,09s \\
      \hline
      6 & 150 & 250 & Roleta & ELIT & 20\% & 98,2\% & 28,19s \\
      \hline
      7 & 100 & 250 & Roleta & ELIT & 25\% & 95,1\% & 13,7s \\
      \hline
      8 & 100 & 200 & TOUR 3 & ELIT & 50\% & 86,5\% & 8,23s \\
      \hline
      9 & 200 & 200 & TOUR 3 & ELIT & 50\% & 94,6\% & 32.16s \\
      \hline
      10 & 200 & 200 & TOUR 5 & MPF & 20\% & 85,5\% & 172,87s \\
      \hline
    \end{tabular}
    \caption{Resultados da segunda etapa}
    \label{tab:etapa2}
  \end{table}

  Ao analisar os resultados apresentados na Tabela \ref{tab:etapa2} percebemos
  que a configuração de número 4 obteve 100\% de taxa de convergência, porém,
  seu tempo de execução é 6,3 vezes o tempo do arranjo 8 que obteve 86,5\%, ou seja,
  pouco aumento de convergência para muito tempo de execução.

  Comparando o resultado 8 com os demais percebe-se que a disposição 2 é apenas
  1,3 vezes mais demorada e sua taxa de convergência é 8\% maior, apresentando assim
  uma boa relação entre tempo de execução e convergência, por isto foi a configuração
  escolhida.
  \subsection{Etapa 3}
  Nesta fase a melhor configuração da Etapa 2, número 2 da Tabela \ref{tab:etapa2},
  foi utilizada para resolver novos problemas, sendo eles:
  \begin{enumerate}
  \item EAT + THAT = APPLE;
  \item CROSS + ROADS = DANGER;
  \item COCA + COLA = OASIS;
  \item DONALD + GERALD = ROBERT.
  \end{enumerate}

Os resultados encontrados não foram satisfatórios e podem ser encontrados na
Tabela \ref{tab:etapa3}.

  \begin{table}[h]
    \centering
    \begin{tabular}{|c|c|c|}
      \hline
      Problema & Convergência & Tempo \\
      \hline
      1 & 26,2\% & 10,53s \\
      \hline
      2 & 1,0\% & 10,56s \\
      \hline
      3 & 9,5\% & 10,34s \\
      \hline
      4 & 5,1\% & 10,31s \\
      \hline
    \end{tabular}
    \caption{Resultados da terceira etapa}
    \label{tab:etapa3}
  \end{table}

A fim de melhorar os resultados testou-se funções de avaliação a seguir:

  \begin{equation}
    Av = (\prod_{i=1}^{N} (1+\mid R_i - E_i\mid)) -1
    \label{eq:avprod}
  \end{equation}
Onde:
  \begin{description}
    \item[\(Av\)] = Avaliação
    \item[\(R_i\)] = i-esima letra do resultado da soma
    \item[\(E_i\)] = i-esima letra do resultado esperado
  \end{description}

Para a eq. \ref{eq:avprod} os resultados encontrados estão na Tabela \ref{tab:avprodt}.

\begin{table}[h]
  \centering
  \begin{tabular}{|c|c|c|}
    \hline
    Problema & Convergência & Tempo \\
    \hline
    1 & 26,2\% & 10,53s \\
    \hline
    2 & 1,0\% & 10,56s \\
    \hline
    3 & 9,5\% & 10,34s \\
    \hline
    4 & 5,1\% & 10,31s \\
    \hline
  \end{tabular}
  \caption{Resultados da terceira etapa com a avaliação 1}
  \label{tab:avprodt}
\end{table}

\end{document}
